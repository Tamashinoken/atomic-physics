\documentclass[epsf]{article}
\usepackage{amssymb}
\usepackage[english]{babel}
\usepackage{tikz}
\usepackage{geometry}
\linespread{1.6}
\usepackage{hyperref}
\usepackage{amsmath}
\geometry{
	paper=a4paper,
	top=3.5cm,
	bottom=2.5cm,
	right=2cm,
	left=3cm
}
\begin{document}
\begin{titlepage}
  \begin{center}
    \large
    O`ZBEKISTON RESPUBLIKASI\\ OLIY VA O`RTA-MAXSUS TA'LIM VAZIRLIGI
        MIRZO ULUG`BEK NOMIDAGI\\O`ZBEKISTON MILLIY UNIVERSITETI\\FIZIKA FAKULTETI
    \vspace{0.25cm}
     
 \vfill
     
    \Large B/19-02 guruh talabasi
    
     
     
    \Huge{Musaxodjayeva Mushtariy Sanjar qizi}
   
 
    \textsc{Kurs ishi}\\[5mm]
     
    {\Huge Atom fizikasi bo`limiga oid mashg`ulotlarni o`qitishda pedagogik texnologiyalardan foydalanish usullari\\
      }
  \bigskip
\end{center}
\vfill

 \huge Qabul qildi: Begmatova Dilfuza
 \vfill


\begin{center}
  Toshkent, 2020
\end{center}
\end{titlepage}
\newpage
\renewcommand{\contentsname}{Mundarija}
\tableofcontents
\newpage
\section{Kirish}
\hspace{0.4cm}
Ta’lim, bir so’z bilan aytganda, mamlakatning kelajakdagi istiqbolini
ta’minlaydigan yosh avlodni tarbiyalash, savodli qilishdir.Turli ilmiy tadqiqot
metodlarini fizika fanini o’qitish jarayonida qo‘llash ta’lim samaradorligini oshiradi,
o‘quvchilarning mustaqil fikrlash jarayonini shakllantiradi, o‘quvchilarda mavzuni
o‘rganishga ishtiyoq va qiziqishni oshiradi, olingan bilimlarni mustahkamlash,
o‘zlashtirish, ulardan amaliyotda erkin foydalanish ko‘nikma va malakalarini
shakllantiradi[6].

O’zbekiston Respublikasi Kadrlar tayyorlash milliy dasturiga muvofiq ta’lim muassasalarini maxsus tayyorlangan pedagogik kadrlar bilan ta’minlash, ularning ish jarayonida raqobatga asoslangan muhitni yuzaga keltirish, o’quv-tarbiya jarayonini sifatli o’quv adabiyotlari bilan va ilg’or pedagogik texnologiyalar bilan ta’minlash kabi masalalalarni amalga oshirishni nazarda tutadi.  Ushbu vazifalarni amalda bajarish har bir ta’lim muassasasining  bevosita burchi hisoblanadi.Ta’lim-tarbiya jarayoniga pedagogik texnologiyalarni muvaffaqiyatli ravishda tadbiq etish har bir fan o’qituvchisining maxsus bilim va ko’nikmalarga hamda pedagogik amaliyotda zarur bo’ladigan metodik tayyorgarlikka ega bo’lishini taqozo etadi.
\subsection{Kurs ishi mavzusining dolzarbligi}
\hspace{0.4cm}
Ta’limning  fan  va  ishlab  chiqarish  bilan  integratsiyasi mexanizmlarini rivojlantirish,  uni  amaliyotga  joriy  etish,  o’qishni,  mustaqil  bilim  olishni individuallashtirish  hamda  masofaviy  ta’lim  tizimi  texnologiyasini,  uning vositalarini  ishlab  chiqish,  o’zlashtirish,  yangi  pedagogik  va  ilg’or texnologiyalari asosida talabalarni o’qitishni jadallashtirish ana shunday dolzarb vazifalar  sirasiga  kiradi.  Ushbu  vazifalarni  bajarish  mavjud  pedagogik jarayonlarni takomillashtirishni,  uni  hozirgi  zamon  talablariga  mos rivojlantirishni, xususan pedagogik ta’lim paradigmasini zamonaviy pedagogik va  ilg’or  texnologiyalarini  o’zlashtirishga,  o’rta  maxsus,  kasb-hunar ta’limi  muassasalarida  kasbiy  tayyorgarligi  yuqori  bo’lgan  pedagog  kadrlarni tayyorlashga yo’naltirishni taqozo etadi[6].

Atom fizikasi fizikaning eng qiziqarli, shu bilan birga o`rganish eng qiyin bo`lgan bo`limi hisoblanadi. Shu sababli ushbu fanni o`qitishning o`ziga xos usullarini ishlab chiqish juda dolzarb masaladir.

\subsection{Kurs ishining maqsadi}
\hspace{0.4cm}
Umumiy o`rta ta'lim maktablarining yuqori sinflarida, akademik  litsey  va  kasb-hunar kollejlarida atom fizikasini o`qitishning turli metodik usullarini    asoslashdan iborat.
\subsection{Kurs ishining  vazifalari.}
\hspace{0.4cm}
Kurs ishida nazariy va  amaliy  isbotni  taqozo  qiladigan  ilmiy  faraz,tadqiqot  ob’yekti,  predmeti, maqsadiga muvofiq quyidagi tadqiqot vazifalari hal qilinadi:
\begin{enumerate}
	\item O’rta  maxsus,  kasb-hunar  ta’limining  fizika  kursida  ilg’or texnologiyalarining umumiy masalalarini tahlil qilish;  
	\item  Ilg’or texnologiyalatining jamiyat taraqqiyotidagi o’rni va o’quv jarayonidagi ahamiyatini o’rganish;
	\item  Fizika  faniga ilg’or  texnologiyalarni  joriy  qilish  usullarini aniqlash;
	\item  O’rta maxsus, kasb-hunar ta’limida zamonaviy texnologiyalar tadqiqoti masalalarini o’rganish.
	
\end{enumerate}
\section{Asosiy qism}
\subsection{Pedagogik texnologiyalarning turlari}
\subsubsection{Illustratsiyalar yordamida tushuntirish}
\hspace{0.4cm}
Illustratsiyalab tushuntirish - bu metod qo'llanganda, o'qituvchi turli
vositalar yordamida o'quvchilarga o'quv materialini tushuntiradi, o'quvchilar
esa, materialni tayyor holda qabul qilishadi hamda tushunishga harakat qilib,
esda saqlab qolishadi. Bu jarayonda o'qituvchi materialning mazmunini
og'zaki bayon qiladi va turli o'qitish vositalaridan foydalanadi hamda darsni
tashkil qilishning turli shakllaridan foydalanadi va mohirlikning namunasini
amalda ko'rsatib beradi. Natijada o'quvchilar bilimlarni o'zlashtirishdagi
birinchi darajali harakatlarni bajarishadi, boshqacha aytganda, ular
o'qituvchining aytganlarini eshitishadi, kitob bilan mustaqil ishlashadi,
jismlarni va ulaming modellarini ko'rishadi va kuzatishadi.
Bunday metod yoshlarga bilim berishning eng unumli yo'llaridan
biridir. Uning samarali ekanligi o'rta va oliy maktablarning ko'p yillik
amaliyotida sinalgan va o'qitishning barcha bosqichlarida foydali, deb
hisoblangan.

Shuningdek, akademik litsey va kasb-hunar kollejlarida ham bu usulda dars o`tish yetarlicha keng tarqalgan. Bu usulda dars o`tilganda o`quvchida mavzu bo`yicha yetarlicha aniq tasavvur hosil bo`lishi uchun hayptiy misollar keltirib o`tilishi ham maqsadga muvofiq. 

Biroq ushbu metodni qo'llab dars o'tganda o'quvchining
faoliyati qabul qilish, tushunish va esda saqlab qolish bilangina cheklanadi. Ular olgan bilimning sifati tekshirilmaydi va uning amalda qo'llanishi shakllantirilmaydi.

\subsubsection{Ikki tomonlama faollik metodi}
\hspace{0.4cm}
Bu metodni qo'llagan paytda, o'qituvchi o‘quvchilarga turli vazifalar
berish bilan, ular egallagan bilimining sifatini tekshiradi. O'quvchilar, o'qituvchining savoliga ko'ra, esida saqlab qolganlarini aytib berishadi, sinfda
yoki auditoriyada o'qituvchi ko'rsatgan masalaga o'xshash masalalarni yechishadi. Berilgan reja bo'yicha insholar, bayonlar va referatlar yozishadi.
Tayyor ko'rsatma bo'yicha fizika va kimyodan tajribalar o'tkazishadi.
O'quv adabiyotida berilgan yoki o`qituvchi ko'rsatgan rasmlarni, grafiklarni
yoki chizmalarni tayyorlashadi va kerakli jadvallarni to'ldirishadi.
Reproduksiyalash
metodining samaradorligini
oshirish
uchun
metodistlar, ayrim ilg'or o'qituvchilarning ko'nikmalar tizimini, didaktik
materiallarni, dasturlangan o'quv qurollarni. tayanch signallarni, konspektlar va bloklarni tuzishadi. Jumladan, V.F.Shatalovning dars berish usuli bunga ochiq misol bo'la oladi. Ushbu metodni qo'llash, o'qitishni algoritmlashtirishga bog'liq. Algoritmlashtirish deganda - o'quvchi va talabalarga o'quv faoliyatini tashkil qilish tartibini hamda rejasini o'rgatishni
tushunamiz. Ular har bir o'quv ishini ushbu algoritmga mos holda bajarishadi. Ammo bu holda ham ularning fikr yuritishi cheklanganligicha qolaveradi, ijodkorlik qobiliyatlari kerakli darajada o`smaydi va rivojlanmaydi. Bunday holda rivojlanish, o`quv materialining muammoli o`qitish metodini qo`llash orqali amalga oshiriladi.

\subsubsection{Muammoli o`qitish}
\hspace{0.4cm}
O’quv materialini muammoli bayon qilishning mazmuni quyidagicha.
O'qituvchi darsda o’quv materialini tushuntirishda, o'quvchi va talabalarning
oldiga kerakli muammolarni qo'yadi va ularni hal qilish yo'llarini ko'rsatib
beradi. Bundan asosiy maqsad - ularga muammoni, muammoli vaziyatning
mazmunini tushuntirish, qanday savollar yoki masalalarni o'quv muammosi
sifatida qarash mumkinligini bildirish, uni hal qilish yo'llarini ko'rsatishdan
iborat bo'ladi. Muammoli bayon qilish o'quvchi va talabalami bilish
jarayonining mantiq va usullari bilan tanishtiradi. Shu bilan birga, ular o'quv
materialini o'zlashtirishga ijodiy yondoshadi.

O'qitish jarayonida muammoli vaziyatni, asosan, ikki yo`l bilan hosil
qilish mumkin.
\begin{enumerate}
	\item 
	Muammoli vaziyat, o'qituvchining maqsadli uyushtirilgan harakatisiz
	ham o'quvchi, ham talabalarning mustaqil ishlashi asosida stixiyali (tartibsiz)
	ravishda paydo bo`ladi. Ular darslik yoki qo'shimcha adabiyotlarni o'qish,
	radiodan eshitish yoki televideniye orqali ko'rganlarini tahlil qilish, masala
	yechish yoki mustaqil eksperiment o'tkazishda, o'qituvchi tomonidan esga
	olinmagan turli muammolarni ko'rishlari mumkuin. Haqiqatan, bunga o‘xshagan holatlar, o'rta va oliy maktablar amaliyotida ko'p uchraydi. Ular
	muammoning hal qilinishini o'z vaqtida o'qituvchilardan so'rashadi, shu
	bilan birga, ayrim hollarda o'zlari taklif qilgan javoblarni ko'rsatishadi.
	Bunday yutuqlami qo'llash va yanada rivojlantirish uchun, ularga bu masala
	bo'yicha to'g'ri maslahat berish kerak.
	
	\item Aksariyat hollarda, muammoli vaziyat o'qituvchi tomonidan maqsadli
	yuzaga keltirilib, muammoni hal qilinishi, uning bevosita rahbarligida
	amalga oshadi. Buning uchun, o'qituvchi tegishli mavzuning mazmuniga
	mos muammoli savollar tizimini ma'lum ketma-ketlikda tuzib chiqadi. Ular
	o'qitishning qaysi bosqichida (yangi materialni tushuntirish yoki o'tilgan
	mavzularni takrorlash paytida), qachon va qayerda (auditoriya yoki uyda)
	bajarishlari, o'quvchilarga qanday shaklda taklif qilishlari aniqlanadi. Albatta,
	bu savollar va vazifalar o'tiluvchi va oldin o'tilgan materialning mazmuniga,
	o'quvchi va talabalarning nazariy va amaliy bilimlari darajasiga, ularning
	qabul qilishi va o'zlashtirish qobiliyatlariga moslab tuziladi.
	
\end{enumerate}
\subsubsection{"Ilmiy munozara" metodi}
\hspace{0.4cm}
Ilmiy munozara metodi (JIGSAW metodikasi) barcha qit'alarning ko'plab mamlakatlaridagi o'quv jarayonida, o'quv dasturining aniq fanlarini o'rganishda ham, turli ijtimoiy tadbirlarda ham qo'llanilgan. Ushbu usul yordamida o`zlashtirishi pastroq bo`lgan o`quvchilar, o`zlarining yaxshi o`qiydigan sinfdoshlari bilan birga ishlash jarayonida turli bilimlarni o`zlashtiradilar [7]. 
Bu talabalarning ijtimoiy va intellektual rivojlanishiga qaratilgan jarayon. Turli tadqiqotlar shuni ko'rsatdiki, ayniqsa boshlang'ich, o'rta va universitet darajalarida "Jigsaw" texnikasi nazariy kurslarda, talabalarning tanqidiy fikrlash jarayonlarini o'z fikrlarini ifoda etish qobiliyati va muloqot qobiliyatlarini rivojlantirishda samarali ekan. 
\subsection{Atom fizikasini o`qitishda "Ilmiy munozara" metodidan foydalanish}
\hspace{0.4cm}
Hamkorlikda o'qitish - bu foydalanish uchun qulay, lekin ko'pincha uni amalga oshirish qiyin bo'lgan pedagogik metodika[2]. Hamkorlikda o'qitishni qo'llashning go'zalligi shundaki, fan o'qituvchisi tushunchalarni mustahkamlashi kerak bo'lgan talabalar uchun yana bir manba sifatida yuqori qobiliyatli talabalardan foydalanishi mumkin. Ushbu atom tuzilishi darsida birgalikda o'qitish atomlarning tartib raqami, massa soni, izotop va atom yadrosi kabi tushunchalarni o'rgatish uchun ishlatiladi. Har bir o'quv guruhi talabalarning kuchli va zaif tomonlarini hisobga olgan holda tuziladi. Qobiliyati yuqori o'quvchilar atom tuzilishi tushunchalarini tengdoshlariga tushuntirishga yordam berishga da'vat etiladi. Chunki o`zlashtirishi pastroq bo`lgan o`quvchilar o`z sinfdoshlaridan eshitgan ma'lumotlarni yazshiroq qabul qilishadi. Darsning borishi taxminan mana bunday ko`rinishda bo`lishi mumkin:
\begin{itemize}
	\item 

O`quvchilarni guruhlarga bo`lish.

Buning uchun o`qituvchi kartochkalardan foydalanishi mumkin. Har bir o`quvchi kartochkalardan bittadan tanlaydi va ko`rsatilgan joyga borib o`tiradi. Bu usulning yagona kamchiligi shundaki, tasodifan o`zlashtirishi qiyin bo`lgan hamma o`quvchilar bir guruhga, nisbatan o`zlashtirishi yaxshiroq bo`lgan o`quvchilar bitta guruhga tushib qolishi mumkin. Lekinn eng asosiysi, bu usul yordamida shakllantirilgan guruhlar tasodifiy ishtirokchilardan iborat bo`ladi. Bu esa o`quvchilarda guruh bilan ishlash ko`nikmasini shakllantiradi. 
\item Guruhlarning vazifalarini taqsimlash.

O`qituvchi o`zlashtirishi nisbatan yaxshi bo`lgan bitta guruhni tanlaydi va ushbu guru a'zolariga quyidagi vazifalarni bo`lib beradi:
\begin{itemize}
\item Atom tartib raqamlari bo'yicha mutaxassis ekspert.
Bu odam guruhning boshqa a'zolariga atomning tartib raqami nimani anglatishini o'rgatish bilan shug'ullanadi. Ular davriy jadval yordamida atom raqami qanday topilishini tushuntiradilar. So`ngra atom tartib raqamini elektronlar va protonlar soni bilan ham bog'lashadi.
\item 
Massa soni bo`yicha mutaxassis. 
Bu odam guruhning boshqa a'zolariga massa soni nimani anglatishini o'rgatish bilan shug'ullanadi. Ular massa sonini qanday topish va uning subatomik zarralar yoki nuklonlar (protonlar va neytronlar) bilan qanday bog'liqligini tushuntiradilar.
\item 
Izotop bo'yicha mutaxass.
Bu odam guruhning boshqa a'zolariga izotop nima ekanligini o'rgatish bilan shug'ullanadi. Ular bitta element izotoplari nimasi bilan bir-biridan  farqlanishini tushuntiradilar. Shuningdek, ular bitta elementning turli izotoplari yadrolarining rasmlarini chizadilar.

\item Atom yadrosi bo`yicha mutaxassis.
Ushbu shaxs guruhning boshqa a'zolariga atom tartib raqami va massa sonini davriy jadvaldagi belgilaridan foydalanib ko'rsatishni o'rgatish uchun javobgardir.
\end{itemize}
Guruhlar sinf rahbari tomonidan aniqlanadi. Ideal holda guruh hajmi 4-6 kishidan iborat bo`lgani ma'qul.
Eslatma. Ushbu mashq uch qismga bo'linadi, ular vaziyatga qarab bir yoki bir nechta mashg'ulotlar davomida bajarilishi mumkin.

Qo'shma o'quv guruhlari tuziladi va o`quvchilar butun faoliyat davomida bir guruh bo'lib ishlaydilar. 1-qism uchun quyidagi ishlar bajariladi:
- Qo'shma o'quv guruhlarini joriy etish.

- O`quvchilarni rollari bilan bir qatorda o'z guruhlariga tayinlash.

- Darslikdagi atom tuzilishi bo'limidan ma'lumotnoma sifatida foydalanish.

- O`quvchilarga dars haqida umumiy ma'lumot berish.

- Matnni ma'lumotnoma sifatida ishlatishdan tashqari, davriy jadvaldan ham o'quv qo'llanma sifatida foydalanish mumkin. Atom va massa sonlari bo'yicha mutaxassislar davriy jadvalda atom sonlari va atom massalarining joylashishini bilishlari kerak. Izotop olimlari atom massasi har xil massa sonlari bo'lgan, ammo bir xil atom raqami bo'lgan atomlarning og'irligi o'rtacha qanday ekanligini bilishlari kerak. Va nihoyat, yadroshunos olimlar davriy jadvalda ma'lumotlarning aks ettirilishi va uni atom yadrosining turli modellari yordamida qanday talqin qilinishi bilan bog'lashlari kerak.

- ekspert guruhlaridagi har bir talabaga o'z ishlarini batafsil bayon qilgan varaqni tarqatish. Ular birgalikda ishlashlari kerak. Ushbu jadvallar bobdagi maqsadlarga mos kelishi va mutaxassislarga o'zlari o'rgatadigan tushunchalarini tushunishning bir usuli sifatida xizmat qilishi kerak.

\item Darsni yakunlash. Xulosalarni chiqarish.

O'qitish usuli sifatida kooperativ o'rganish atom tuzilishi atrofidagi tushunchalarni o'rgatish uchun juda samarali bo'ladi. Har bir talaba bitta tushunchaning mutaxassisi bo'lishi kerakligi sababli, hech kim o'zini passiv tuta olmaydi. Guruhlar shunday tashkil etilganki, har bir guruhda yetakchilik qobiliyatiga ega bo'lgan bitta talaba bo'lishi kerak edi.

Aksariyat hollarda fizika talabalari ushbu hamkorlik darsini yaxshi o'tkazadilar. O'qituvchining darsini tinglash o'rniga guruh ishlarini bajarishga qodir bo'lgan umumiy ishtiyoq ular uchun qiziqarliroq tuyuladi. O'quvchilarning har qanday guruhida bo'lgani kabi, ko'pincha guruhni boshqarish vazifasini bajarish qiyin. 
\end{itemize}

Dars oxirida esa, olingan bilimlarni tekshirish uchun quyidagi kichik topshiriqdan foydalanish mumkin:

\renewcommand{\arraystretch}{1.3}
\fbox{
\begin{minipage}[t]{\textwidth}

\begin{center}
	\textbf{Mustahkamlash uchun savollar}
\end{center}
\textit{Bo`sh joylarga quyidagi keltirilgan so`zlarni qo`yib chiqing. Har bir so`z faqat bir marta ishlatilishini unutmang!}

Atomning tartib raqami, Massa soni,  Protonlar, 
Elektronlar Izotoplar,  Neytronlar

\begin{enumerate}
\item  Atomning tartib raqami atomda \_\_\_\_\_\_\_\_\_\_\_\_ va \_\_\_\_\_\_\_\_\_\_\_ qancha ekanligini bildiradi.
\item Davriy jadvaldagi to`q shriftda yozilgan raqam \_\_\_\_\_\_\_\_\_\_\_\_\_\_\_ deb nomlanadi. Bu har bir element uchun o'ziga xosdir.
\item  Atomdagi proton va neytronlarning umumiy soni  \_\_\_\_\_\_\_\_\_\_\_\_\_\_\_\_\_\_\_\ deyiladi.
\item  Protonlari bir xil, ammo neytronlari turlicha bo'lgan atomlarni \_\_\_\_\_\_\_\_\_\_\_\_\_ deyiladi.
\item  Zaryadga ega bo`lmagan  zarralar \_\_\_\_\_\_\_\_\_\_\_\_\_\_\_ deyiladi.
\end{enumerate}

\begin{center}\textbf{Quyidagi jadvalni to`ldiring}:

\begin{tabular}{|c|c|c|c|c|c|}
	\hline
Tartib raqami& Massa soni&$N_{p}$&$N_{n}$&$N_{e}$&Kimyoviy belgisi\\
\hline
6&13& & & & \\
\hline
&25&	12&&&\\ \hline
&	175& &&			84&\\ \hline
13&&		&	15&& \\ 
\hline
\end{tabular}
\end{center}
\end{minipage}
}
\newpage
\section{Xulosa}
\hspace{0.4cm}
Tadqiqotlarning[1,2,3] ko`rsatishicha, o`quvchilar o`zlarining do`stlari bilan, o`qituvchiga qaraganda ancha aktivroq muloqotga kirishishadi. Ayniqsa birgalikda o`rganish jarayonida ular yangi materialni 80-90\% gacha o`zlashtirishi aniqlangan[4].

Birgalikda o`rganish metodi - fizikani o`rganish uchun juda mos keladi. Ushbu metodni, ayniqsa, laboratoriya mashg`ulotlarida qo`llash juda yaxshi natijalarni beradi. Agar har bir mayda detallariga e'tibor bergan holda tayyorlansa, yuqori sinf o`quvchilari yoki akademik litsey va kasb-hunar kolleji talabalari uchun ham yaxshi samara beradi. Nazariy jihatdan ushbu metod quyidagi afzalliklarga ega:
\begin{itemize}
	\item o`quvchilarning o`ziga bo`lgan ishonchini oshiradi;
	\item guruh bilan ishlash ko`nikmalarini shakllantiradi;
	\item o`z fikrini isbotlash uchun argumentlar ko`rsatishni o`rgatadi;
	\item shaxsiy va jamoaviy mas'uliyatni rivojlantiradi;
	\item o`zi o`rgangan bilimlarni boshqalarga o`rgatish orqali o`quvchilar bilimi mustahkamlanadi
	\item tanqidiy, mantiqiy va eng asosiysi, mustaqil fikrlashni rivojlantiradi. 
\end{itemize}

Guruhlarga bo`lib o`rganish metodini atom fizikasi bo`yicha dars mashg`ulotlarida ham tatbiq etish yuqori natijalarni ko`rsatadi.
 
\newpage

\section{Foydalanilgan adabiyotlar}
\begin{enumerate}
	\item Stefania Elbanowska-Ciemuchowska, "How to Stimulate Students’ Interest in Nuclear Physics?", US-China Education Review, ISSN 1548-6613
	\item Gelu Maftei, "The strengthen knowledge of atomic physics using the “mosaic” 	method (The Jigsaw method)", Procedia Social and Behavioral Sciences 15 (2011) 1605–1610
	\item Juergen Petri and Hans Niedderer, "A learning pathway in high-school level quantum atomic
	physics", International Journal of Science Education, November 1998
	\item Hillery V.A., "Teaching atomic and nuclear physics in high school", Oklahoma State University
	Stillwater, Oklahoma
	1956
	\item G.Maftei, F.F. Popescu, "Teaching atomic physics in secondary school with the jigsaw technique", Romanian Reports in Physics, Vol. 64, No. 4, P. 1109–1118, 2012
	\item Tursunov I., Nishonaliyev U. Pedagogika kursi. - Toshkent:"O`qituvchi", 1997.
	\item Elisabeth Netzell, "Using models and representations in learning and teaching about the atom", Examensarbete inom Fysik, forsknings-konsumtion, grundl $\ddot a$ggandeniv$\dot a$, 15 hp93XFY1
	\item Internet saytlari:
	
 \href{https://consiliumeducation.com/itm/2019/06/22/understanding-atoms/}{https://consiliumeducation.com/itm/2019/06/22/understanding-atoms/}
 
 \href{https://pumas.nasa.gov/examples/index.php?id=80}{https://pumas.nasa.gov/examples/index.php?id=80}
 
 \href{https://cutt.ly/Xh5gxyk}{https://cutt.ly/Xh5gxyk}
\end{enumerate}
\end{document}
